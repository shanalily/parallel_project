\documentclass[10pt,a4paper]{article}
\usepackage[hyperref]{acl2019}
\usepackage{times}
\usepackage{latexsym}
\usepackage{graphicx}

\usepackage{url}

\aclfinalcopy % Uncomment this line for the final submission
%\def\aclpaperid{***} %  Enter the acl Paper ID here

%\setlength\titlebox{5cm}
% You can expand the titlebox if you need extra space
% to show all the authors. Please do not make the titlebox
% smaller than 5cm (the original size); we will check this
% in the camera-ready version and ask you to change it back.

\newcommand\BibTeX{B\textsc{ib}\TeX}

\title{Parallel Programming CSCI 4320 Project Report: \\
City Traffic Discrete Event Simulation}
\author{Alex Garten \\
  RPI \\
  \texttt{gartea@rpi.edu} \\\And
  Shoshana Malfatto \\
  RPI \\
  \texttt{malfas@rpi.edu} \\}

\date{}

\begin{document}
\maketitle
\begin{abstract}
    We implemented a parallel discrete event simulator for traffic patterns at intersections using C and MPI.
\end{abstract}

\section{Introduction}

Our goal was to implement a parallel discrete event simulator which models vehicle traffic patterns to reveal which traffic rules result in the shortest trips between a location A and location B, in a large grid modeled after a city. Though implementations of traffic pattern DES already exist, we will use this as an opportunity to learn how to create a DES, since they have many applications.

\section{Background}
Our knowledge of parallel discrete event simulators (PDES) comes originally from the course lecture. Parallel DES help represent large-scale systems that are hard to understand, and which may require long run times if represented in real-time. A parallel DES should significantly reduce the run time. For instance, a simulation of air traffic could take a few minutes instead of a several years by looking at only specific events that are deemed important, and looking at them in parallel when possible. This could be applied to many other topics like communication networks or biology. An old article \cite{Fujimoto:1990:PDE:84537.84545} that contains some of the information from class provided a good overview of what discrete event simulators can consist of.

In class we talked about Rensselaer’s Optimistic Simulation System (ROSS) \cite{Barnes:2013:WSE:2486092.2486134} which runs Time Warp, a protocol that increases parallelism through rollback and anti-message mechanisms. This protocol scales very well in performance tests on RPI's Blue Gene/Q. We could have chosen to use ROSS to build our implementation, or looked into OMNeT++ \cite{Varga:2008:OOS:1416222.1416290}, which a popular open-source discrete event simulation environment, but we decided to create our simulator from scratch.

\section{Implementation}


We started with an implementation with a more complex grid, but started having memory errors close to the deadline which we would have struggled to find in time for submission, so we went with a simpler grid for our second implementation. However, we will go over the initial implementation since that was our idea going into this project and we spent time writing it, and because we kept many of the plans for this first implementation were maintained in the second implementation.

\subsection{Implementation 1}

We used a conservative approach to build our parallel DES, where cars can stop at a discrete number of locations, and at most one movement could occur within a tick. In our large city, which was a grid of streets going either north-south or east-west, cars only needed to know about the cars near them, so traffic could run somewhat independently in different regions. However, synchronization is required to accurately move cars across the borders of their region.

For parallelization we used MPI, which we preferred to relying on threads after seeing the performance of both in Assignment 4/5. We used a side length for our grid of 32,768 blocks. For the full number of rows, this was multiplied by 2 so that even rows correspond to global coordinates of east-west streets, and odd rows correspond to global coordinates of north-south streets. The rows get divided up by the number of ranks, so that there would need to be communication with "ghost rows" like in homework 4/5.

\begin{figure}
    \centering
    \includegraphics[scale=0.4]{parallel.jpg}
    \caption{Original street design, with  streets (where you consider a street as just one block). The global index is displayed for some of the streets. Even row indices correspond to horizontal streets, and odd row indices correspond to vertical streets.}
    \label{fig:my_label}
\end{figure}

Each street was a C struct that contained an array for the lane going in one direction (east or south) and the lane going in the opposite direction (west or north). Each lane could hold 4 cars at one time. This resembled a queue, which is a commonly seen data structure in discrete event simulators.

There are intersection structs for each place where four streets should meet. This made moving cars cleaner for the programmer to implement and read.

\begin{figure}
    \centering
    \includegraphics[scale=0.6]{parallel2.jpg}
    \caption{An intersection, including street indices from initial implementation (when each lane on a block had four locations). There can be at most 4 cars going through an intersection at a time.}
    \label{fig:my_label}
\end{figure}

The beginning of the simulation started with cars being randomly generated at locations on the grid, with random destinations. These locations are represented by a row index, a column index, and a street index which corresponds the eastern or southern direction.

At each step in time, or 'tick', ghost rows were sent between ranks. This required sending arrays of car objects, which are not an MPI\_Datatype.

Then intersections were updated according to the specified rules. Cars tried to turn in a direction that would get them closer to their destination on the global grid, but they must follow the rules of the road and not crash into other cars. For instance, if the car at the northern side of the intersection goes straight south, the car on the eastern side of the intersection can't go any direction but right within that tick of time. Then the streets would be updated by moving cars along them. Each car was supposed to move at most one grid location per tick.

Our program worked with only 1 rank, but when multiple ranks were used, the program would terminate unexpectedly. There were so many places that memory could be used incorrectly, and parallel programs are difficult to debug, so after some time spent trying to discover the bug, we changed plans to a different implementation of the city traffic problem.

\subsection{Implementation 2}

\begin{figure*}
    \centering
    \includegraphics[scale=0.5]{imp2_diag.png}
    \caption{An model of the graph that implimentation 2 was supposed to solve. Here the blue car starting at 0,0 goal is to get to 1,0 and it does that at t=1. The orange car starts at 1,1 and its goal is to get to 0,0. It does that at t=2.}
    \label{fig:my_label}
\end{figure*}

After the faluire of our first try we attempted to simplify and refocus our simulation. We kept many of the same details from the initial implementation,but now there are only intersections and no streets. This reduced complexicity by eliminating much of the computation. We no longer had to manage street arrays or have complicated intersections traffic rules. The goal of the this implimentation was to refocus it on communication between MPI ranks. This refocus was also its downfall. 

In our first implimentation although the traffic mechanics were complicated, the MPI exchange was actually quite simple. All data could be sent before each tick and the implimentation was parallel deterministic. However, due to some fundamentals the 2nd implimentation was not possible to be made parallel deterministic and required some difficult inter-rank communication. When updating an intersection, the intersection would have to check with its neighboring intersection to see if that intersection was already full in the current tick or in the previous tick. This was a trivial class within one mpi rank. This was not a trivial task between mpi ranks. To do this we developed a bit of an exchange protocol. The requesting rank would send a request with the data it wished to send. The answering rank would receive that data and if it was an eligible move, it would impliment the changes specified and send back a positive confirmation. If it was not an eligible move it would return a negative confirmation. With a positive confirmation the original rank would not continue any further as the move was completed by the other rank. With a negative confirmation the original rank would attempt to proceed knowing that the rank transitory move was not valid.
The additional difficulty of this implimentation in order to keep it highly parallel was that the computation of own data needed to happen in parallel to the MPI communication. Although there
were checks in place to make sure each rank answered all of its corresponding requests before moving,
the system got deadlocked (some of the time with frequency scaling up with size of grid and number of ranks) with more than 2 ranks. We are not sure exactly what happened, but it is likely either some element of our protocol was incorrect or the checks got corrupted with an errounous send or receive
deadlocking the system.
This difficulty in protocol was magnified by the graph like nature of our problem. Each intersection at a point in time could be considered a node. The movement from a node to another at each tick could be considered an edge. To solve this graph the goal would be to find a non-overlapping path from each starting node to each ending node (technically a set of ending nodes as the time stamp of the node does not matter). The difficulty in solving graph based problems in parallel is well documented due to their incosistant and irregular nature. We definetly encountered this problem in this implimentation as receives had to be posted that never got sent to (and had to be resolved later which is likely where our deadlock issue was).
Given more time it is likely we could've perfected this implimentation either through debugging or the adding of additional inefficiencies to ensure everything worked. Hello




\subsection{Final Implementation}



\section{Performance Results}

\section{Analysis of Performance}
\section{Team Organization}

In our first version, Alex wrote all of the MPI code and the initial architecture, and Shoshana was working on updating intersections and streets according to traffic rules. Alex wrote most of the second version and Shoshana wrote most of the 3rd version. Both wrote the report. Both team members ran jobs on the Blue Gene and debugged.

\section{Conclusion}

% \bibliographystyle{acl2019}
\bibliography{report}


\end{document}
